\documentclass{article}
    \title{\textbf{2a. KMITOČTOVÁ ZÁVISLOST STŘÍDAVÝCH VOLTMETRŮ}}
    \author{Tomáš Kysela}
    \date{14/3/2022}

    \addtolength{\topmargin}{-3cm}
    \addtolength{\textheight}{3cm}

\usepackage[czech]{babel}
\usepackage{graphicx}
\usepackage{circuitikz}
\usepackage{amsmath}
\usepackage{subcaption}
\usepackage{pgfplots}
\usepackage{siunitx}
\usepackage{multirow}
\usepackage{pgfplots}
\usepackage{tabularx}
\sisetup{detect-all}

\makeatletter
\providecommand\add@text{}
\newcommand\tagaddtext[1]{%
    \gdef\add@text{#1\gdef\add@text{}}}%
\renewcommand\tagform@[1]{%
    \maketag@@@{\llap{\add@text\quad}(\ignorespaces#1\unskip\@@italiccorr)}%
}
\makeatother

\begin{document}

\maketitle

\section{Úkol měření}
\begin{enumerate}
	\item V rozsahu kmitočtů 70 \si{\hertz} až 300 \si{\kilo\hertz} (pro kmitočty 70, 200, 500 \si{\hertz}, 1, 3, 10, 20, 50, 100, 200, 300 \si{\kilo\hertz}) změřte kmitočtovou závislost předložených číslicových voltmetrů. Za kmitočtově nezávislý považujte v tomto frekvenčním rozsahu číslicový voltmetr HP 34401A. Měření proveďte na příslušných rozsazích voltmetrů při hodnotách napětí 1 \si{\volt} a 7 \si{\volt}.
	\item Změřené závislosti vyneste do grafů a teoreticky zdůvodněte.
\end{enumerate}
\section{Schéma zapojení}
\begin{figure}[h]
	\centering
	\begin{circuitikz}
		\draw (0,-0.0) to[twoport,t=$G$] (0,-2.0);
		\draw (2,-0.0) to[rmeter,t=$V_1$] (2,-2.0);
		\draw (4,-0.0) to[rmeter,t=$V_2$] (4,-2.0);
		\draw (6,-0.0) to[rmeter,t=$V_3$] (6,-2.0);
		\draw (8,-0.0) to[rmeter,t=$V_4$] (8,-2.0);
		\draw (0,-0.0) to[short] (2,-0.0);
		\draw (2,-0.0) to[short] (4,-0.0);
		\draw (4,-0.0) to[short] (6,-0.0);
		\draw (6,-0.0) to[short] (8,-0.0);
		\draw (8,-2.0) to[short] (6,-2.0);
		\draw (6,-2.0) to[short] (4,-2.0);
		\draw (4,-2.0) to[short] (2,-2.0);
		\draw (2,-2.0) to[short] (0,-2.0);
	\end{circuitikz}
\end{figure}
\section{Soupis použitých přístrojů}
\begin{tabular}{ll}
	$G$ & RC generátor \\
	$V_1$ & číslicový multimetr HP 34401A, rozsah 1 V a 10 V \\
	$V_2$ & číslicový multimetr Summit 45, rozsah 4 V a 40 V \\
	$V_3$ & číslicový multimetr MY-64 (Mastech), rozsah 2 V, 20 V a 200 V \\
	$V_4$ & nízkofrekvenční elektronický voltmetr U1241C, rozsah 1 V a 10 V \\
\end{tabular}
\section{Teoretický základ}
Číslicové měřící přístroje jsou kmitočtově závislé. Jejich šířka pásma (kde měří správně) se odvíjí od mezního kmitočtu, tj. bodu, kde přenos klesne o 3 dB. U levných kmitočtů často dochází k rezonanci, tj. překmitu. Mezní kmitočet je proto často určen až za překmitem, šířka pásma je větší a tudíž nesprávná.

\section{Naměřené hodnoty}

\subsection{1 V}
\begin{tabular}{c || c | c | c | c}
	\multirow{2}{*}{Kmitočet[\si{\kilo\hertz}]} & \multicolumn{4}{c}{Napětí [\si{\volt}] }\\ & 34401A & SM45 & MY-64 & U1241C \\\hline \hline
	0.07 & 1.016 & 1.010 & 1.020 & 1.016 \\\hline
	0.2 & 1.009 & 1.003 & 1.015 & 1.010 \\\hline
	0.5 & 1.008 & 0.997 & 1.014 & 1.009 \\\hline
	1 & 1.007 & 0.985 & 1.014 & 1.009 \\\hline
	3 & 1.006 & 0.893 & 1.013 & 1.006 \\\hline
	10 & 1.002 & 0.498 & 0.981 & 1.003 \\\hline
	20 & 1.000 & 0.245 & 0.918 & 1.002 \\\hline
	50 & 0.999 & 0.037 & 0.734 & 1.040 \\\hline
	100 & 1.010 & 0.003 & 0.364 & 1.155 \\\hline
	200 & 0.999 & 0.001 & 0.008 & 1.162 \\\hline
	300 & 1.009 & 0.000 & 0.006 & 0.806
\end{tabular}

\begin{tikzpicture}
	\begin{axis}[
		xmode=log,
		log ticks with fixed point,
		xlabel={Kmitočet [\si{\kilo\hertz}]},
		ylabel={Napětí [\si{\volt}]},
		xmin=0, xmax=320,
		ymin=0, ymax=1.2,
		legend pos=south west,
		xmajorgrids=true,
		ymajorgrids=true,
		grid style=dotted,
		]
		
		\addplot[
		color=blue,
		mark=*,
		]
		table {
			0.07 1.016
			0.2 1.009
			0.5 1.008
			1 1.007
			3 1.006
			10 1.002
			20 1.000
			50 0.999
			100 1.010
			200 0.999
			300 1.009
		};
	
		\addplot[
		color=red,
		mark=*,
		]
		table {
			0.07 1.010
			0.2 1.003
			0.5 0.997
			1 0.985
			3 0.893
			10 0.498
			20 0.245
			50 0.037
			100 0.003
			200 0.001
			300 0.000
		};
	
		\addplot[
		color=green,
		mark=*,
		]
		table {
			0.07 1.020
			0.2 1.015
			0.5 1.014
			1 1.014
			3 1.013
			10 0.981
			20 0.918
			50 0.734
			100 0.364
			200 0.008
			300 0.006
		};
	
		\addplot[
		color=magenta,
		mark=*,
		]
		table {
			0.07 1.016
			0.2 1.010
			0.5 1.009
			1 1.007
			3 1.006
			10 1.003
			20 1.002
			50 1.040
			100 1.155
			200 1.162
			300 0.806
		};
		
		\legend{34401A,SM45,MY-64,U1241C}
		
	\end{axis}
\end{tikzpicture}

\subsection{7 V}

\begin{tabular}[]{c || c | c | c | c}
	\multirow{2}{*}{Kmitočet[\si{\kilo\hertz}]} & \multicolumn{4}{c}{Napětí [\si{\volt}] }\\ & 34401A & SM45 & MY-64 & U1241C \\\hline \hline
	0.07 & 7.052 & 6.96 & 7.07 & 7.062 \\\hline
	0.2 & 7.056 & 6.96 & 7.08 & 7.068 \\\hline
	0.5 & 7.046 & 6.95 & 7.07 & 7.059 \\\hline
	1 & 7.046 & 6.93 & 7.07 & 7.053 \\\hline
	3 & 7.048 & 6.87 & 7.21 & 7.042 \\\hline
	10 & 7.028 & 6.62 & 8.95 & 7.034 \\\hline
	20 & 7.025 & 6.26 & 13.43 & 7.022 \\\hline
	50 & 7.034 & 5.25 & 16.3 & 7.050 \\\hline
	100 & 7.017 & 3.62 & 20.5 & 7.145 \\\hline
	200 & 7.018 & 0.63 & 5.7 & 7.603 \\\hline
	300 & 7.002 & 0.14 & 0.9 & 8.061
\end{tabular}

\begin{tikzpicture}
	\begin{axis}[
		xmode=log,
		ymode=linear,
		log ticks with fixed point,
		xlabel={Kmitočet [\si{\kilo\hertz}]},
		ylabel={Napětí [\si{\volt}]},
		xmin=0, xmax=320,
		ymin=0, ymax=21,
		legend pos=north west,
		xmajorgrids=true,
		ymajorgrids=true,
		grid style=dotted,
		]
		
		\addplot[
		color=blue,
		mark=*,
		]
		table {
			0.07 7.052
			0.2 7.056
			0.5 7.046
			1 7.046
			3 7.048
			10 7.028
			20 7.025
			50 7.034
			100 7.017
			200 7.018
			300 7.002
		};
		
		\addplot[
		color=red,
		mark=*,
		]
		table {
			0.07 6.96
			0.2 6.96
			0.5 6.85
			1 6.93
			3 6.87
			10 6.62
			20 6.26
			50 5.25
			100 3.62
			200 0.63
			300 0.14
		};
		
		\addplot[
		color=green,
		mark=*,
		]
		table {
			0.07 7.07
			0.2 7.08
			0.5 7.07
			1 7.07
			3 7.21
			10 8.95
			20 13.43
			50 16.3
			100 20.5
			200 5.7
			300 0.9
		};
		
		\addplot[
		color=magenta,
		mark=*,
		]
		table {
			0.07 7.062
			0.2 7.068
			0.5 7.059
			1 7.053
			3 7.042
			10 7.034
			20 7.022
			50 7.050
			100 7.145
			200 7.603
			300 8.061
		};
		
		\legend{34401A,SM45,MY-64,U1241C}
		
	\end{axis}
\end{tikzpicture}

\section{Závěr}
Dle očekávání byl přístroj 33401A přesný a stabilní na celém rozsahu. Ruční multimetry naopak byly překvapivě stabilní i po rozsazích udávaných výrobcem.

\end{document}
