\documentclass{article}
    \title{\textbf{2a. KMITOČTOVÁ ZÁVISLOST STŘÍDAVÝCH VOLTMETRŮ}}
    \author{Tomáš Kysela}
    \date{14/3/2022}

    \addtolength{\topmargin}{-3cm}
    \addtolength{\textheight}{3cm}

\usepackage[czech]{babel}
\usepackage{graphicx}
\usepackage{circuitikz}
\usepackage{amsmath}
\usepackage{subcaption}
\usepackage{pgfplots}
\usepackage{siunitx}
\sisetup{detect-all}

\makeatletter
\providecommand\add@text{}
\newcommand\tagaddtext[1]{%
    \gdef\add@text{#1\gdef\add@text{}}}%
\renewcommand\tagform@[1]{%
    \maketag@@@{\llap{\add@text\quad}(\ignorespaces#1\unskip\@@italiccorr)}%
}
\makeatother


\begin{document}

\maketitle

\section{Úkol měření}
\begin{enumerate}
	\item V rozsahu kmitočtů 70 \si{\hertz} až 300 \si{\kilo\hertz} (pro kmitočty 70, 200, 500 \si{\hertz}, 1, 3, 10, 20, 50, 100, 200, 300 \si{\kilo\hertz}) změřte kmitočtovou závislost předložených číslicových voltmetrů. Za kmitočtově nezávislý považujte v tomto frekvenčním rozsahu číslicový voltmetr HP 34401A. Měření proveďte na příslušných rozsazích voltmetrů při hodnotách napětí 1 \si{\volt} a 7 \si{\volt}.
	\item Změřené závislosti vyneste do grafů a teoreticky zdůvodněte.
\end{enumerate}
\section{Schéma zapojení}
\begin{figure}[h]
	\centering
	\begin{circuitikz}
		\draw (0,-0.0) to[twoport,t=$G$] (0,-2.0);
		\draw (2,-0.0) to[rmeter,t=$V_1$] (2,-2.0);
		\draw (4,-0.0) to[rmeter,t=$V_2$] (4,-2.0);
		\draw (6,-0.0) to[rmeter,t=$V_3$] (6,-2.0);
		\draw (8,-0.0) to[rmeter,t=$V_4$] (8,-2.0);
		\draw (0,-0.0) to[short] (2,-0.0);
		\draw (2,-0.0) to[short] (4,-0.0);
		\draw (4,-0.0) to[short] (6,-0.0);
		\draw (6,-0.0) to[short] (8,-0.0);
		\draw (8,-2.0) to[short] (6,-2.0);
		\draw (6,-2.0) to[short] (4,-2.0);
		\draw (4,-2.0) to[short] (2,-2.0);
		\draw (2,-2.0) to[short] (0,-2.0);
	\end{circuitikz}
\end{figure}
\section{Soupis použitých přístrojů}
\begin{tabular}{ll}
	$G$ & RC generátor \\
	$V_1$ & číslicový multimetr HP 34401A, rozsah 1 V a 10 V \\
	$V_2$ & číslicový multimetr Summit 45, rozsah 4 V a 40 V \\
	$V_3$ & číslicový multimetr MY-64 (Mastech), rozsah 2 V a 20 V \\
	$V_4$ & nízkofrekvenční elektronický voltmetr Troneer, rozsah 1 V a 10 V \\
\end{tabular}
\section{Teoretický základ}
Číslicové měřící přístroje jsou kmitočtově závislé. Jejich šířka pásma (kde měří správně) se odvíjí od mezního kmitočtu, tj. bodu, kde přenos klesne o 3 dB. U levných kmitočtů často dochází k rezonanci, tj. překmitu. Mezní kmitočet je proto často určen až za překmitem, šířka pásma je větší a tudíž nesprávná.
\end{document}
