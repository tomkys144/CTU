\documentclass{article}
\title{\textbf{4. MĚŘENÍ MALÝCH PROUDŮ}}
\author{Tomáš Kysela}
\date{28/2/2022}

\addtolength{\topmargin}{-3cm}
\addtolength{\textheight}{3cm}

\usepackage[czech]{babel}
\usepackage{graphicx}
\usepackage{circuitikz}
\usepackage{amsmath}
\usepackage{subcaption}
\usepackage{pgfplots}
\usepackage{siunitx}
\usepackage{float}
\usepackage{multirow}
\sisetup{detect-all}

\makeatletter
\providecommand\add@text{}
\newcommand\tagaddtext[1]{%
	\gdef\add@text{#1\gdef\add@text{}}}%
\renewcommand\tagform@[1]{%
	\maketag@@@{\llap{\add@text\quad}(\ignorespaces#1\unskip\@@italiccorr)}%
}
\makeatother


\begin{document}
	
	\maketitle
	
	\section{Úkol měření}
	\begin{enumerate}
		\item V zapojení podle obr. 1 změřte proud germaniovou diodou v propustném  směru v oblasti malých napětí (20 až 100 mV) v pěti bodech charakteristiky:
		\begin{enumerate}
			\item analogovým mikroampérmetrem
			\item číslicovým mikroampérmetrem na různých rozsazích
			\item pomocí převodníku proud - napětí s operačním zesilovačem, u něhož před měřením určete velikost odporu zpětnovazebního rezistoru R tak, aby převod proud - napětí byl $10^{-5} \si{\ampere\per\volt}$
		\end{enumerate}
		Naměřené hodnoty vyneste do společného grafu.
		\item Při měření dle 1a) a 1b) určete chybu metody způsobenou vnitřním odporem ampérmetru.
		\item Z naměřených hodnot určete \textbf{vnitřní odpory použitých mikroampérmetrů}.
	\end{enumerate}
	
	\section{Schéma zapojení}
	\begin{figure}[H]
		\centering
		\begin{subfigure}{0.6\textwidth}
			\begin{circuitikz}[european]
				\draw (1,-0.0) node[ocirc,label=left:$+$]{};
				\draw (1,-2.0) node[ocirc,label=left:$-$]{};
				\draw (2,-0.0) to[rmeter,t=$V_1$] (2,-2.0);
				\draw (4,-0.0) to[D,l=$D$] (6,-0.0);
				\draw (2,-0.0) to[R,l=$R_1$,a=\SI{90}{\ohm}] (4,-0.0);
				\draw (4,-0.0) to[R,l=$R_2$,a=\SI{10}{\ohm}] (4,-2.0);
				\draw (6,-0.0) node[ocirc,label=right:$A$]{};
				\draw (6,-2.0) node[ocirc,label=right:$B$]{};
				\draw (6,-2.0) -- (4,-2.0);
				\draw (4,-2.0) -- (2,-2.0);
				\draw (2,-2.0) -- (1,-2.0);
				\draw (1,-0.0) -- (2,-0.0);
				\draw [dashed] (6,0)  -- (6,-2);
			\end{circuitikz}
		\caption{}
	\end{subfigure}		
	\begin{subfigure}{0.4\textwidth}
		\begin{circuitikz}[european]
				\draw (10,0) -- (8,0);
				\draw (10,0) to[rmeter,t=$\mu A$] (10,-2);
				\draw (10,-2) -- (8,-2);
				\draw (8,0) node[currarrow,label=left:$A'$,rotate=180]{};
				\draw (8,-2.0) node[currarrow,label=left:$B'$,rotate=180]{};
			\end{circuitikz}
			\caption{}
		\end{subfigure}
		\begin{subfigure}{0.4\textwidth}
			\begin{circuitikz}[european]
				\draw (0,-0.0) node[op amp] (opamp) {};
				\draw (opamp.-) -- (-1.5,0.5);
				\draw (opamp.+) -- (-1.5,-0.5) -- (-1.5,-1.5);
				\draw (-1.5,0.5) -- (-1.5,1.5) -- (-1,1.5);
				\draw (-1,1.5) to[R, l_=$R$] (1,1.5);
				\draw (1,1.5) -- (1.5,1.5) -- (1.5,0);
				\draw (opamp.out) -- (1.5, 0);
				\draw (1.5,0) to[rmeter,t=$V_2$] (1.5,-1.5);
				\draw (1.5,-1.5) -- (-2.5,-1.5) node[currarrow,label=left:$B''$,rotate=180]{};
				\draw (-1.5,0.5) -- (-2.5,0.5) node[currarrow,label=left:$A''$,rotate=180]{};
			\end{circuitikz}
			\caption{}
		\end{subfigure}
		\caption{Zapojení pro měření malých proudů}
	\end{figure}
	\section{Soupis použitých přístrojů}
	\begin{tabular}{ll}
		$V_2$ & voltmetr číslicový, typ U3401A, přesnost 0.02\%+4  \\
		$\mu A_1$ & mikroampérmetr magnetoelektrický, tř.přes. 0.5, rozsah 15\si{\micro\ampere}  \\
		$\mu A_2$ & mikroampérmetr magnetoelektrický, tř.přes. 0.5, rozsah 30\si{\micro\ampere}  \\
		$\mu A_C$ & mikroampérmetr číslicový, typ U3401A, přesnost 0.05\%+5  \\
		$R$ & přesný rezistor nebo odporová dekáda, přesnost 0,1 \%  \\
		Př1 & přípravek s odporovým děličem a polovodičovou diodou \\
		Př2 & přípravek s operačním zesilovačem \\
		$U_1$ & zdroj proměnného stejnosměrného napětí s číslicově nastavitelnou hodnotou  \\
		$NZ$ & napájecí zdroj pro OZ  \\
	\end{tabular}
	\section{Teoretický základ}
	Dioda je napájena ze zdroje napětí s odporovým děličem 10 : 1 (obr. 4.1 a)). Zapojíme-li do série s diodou mikroampérmetr, vznikne na něm úbytek napětí srovnatelný s napětím na diodě. Lze též říci, že napětí na rezistoru 10 \si{\ohm} nebude napětím na 	diodě, ale rozdělí se mezi diodu a mikroampérmetr v poměru velikostí jejich odporů, nebo že 	proud  procházející  diodou  v  sérii  s  odporem mikroampérmetru  bude  při  stejném  napětí  na rezistoru 10 \si{\ohm} menší, než je jeho hodnota při spojení bodů A - B nakrátko.
	
	V  zapojení  podle  obr. 1c),  což  je  převodník  proud-napětí  s  operačním  zesilovačem,  měříme 
	velikost proudu diodou skutečně při napětí na diodě daném napětím na odporu R2, neboť mezi 
	vstupy  OZ  (při  zanedbání  napěťového  offsetu)  je  nulový  rozdíl  napětí  (virtuální  nula).  
	Velikost proudu diodou je dána výrazem:
	\begin{equation}
		I=\frac{U_2}{R}
	\end{equation}
	Přídavnou  chybu  měření  může  způsobit  vstupní  klidový  proud  $I_0$  operačního  zesilovače  (pro  
	OP07 platí $I_{0max} = 7\si{\nano\ampere} $); výsledná standardní nejistota měření proudu je pak dána vztahem:
	\begin{equation}
		u_I = \sqrt{(\frac{-1}{R} u_{U_2})^2+(\frac{U_2}{R^2}u_R)^2+(\frac{I_0}{\sqrt{3}})^2}
	\end{equation}
	Absolutní chybu metody p ři měření analogovým nebo číslicovým mikroampérmetrem 
	určíme ze vztahu:
	\begin{equation}
		\Delta_{met}=N-S \tagaddtext{[\si{\milli\ampere}]}
	\end{equation}
	Relativní chybu metody určíme ze vztahu:
	\begin{equation}
		\sigma_{met}=\frac{N-S}{S} \cdot 100\%
	\end{equation}
	kde za naměřenou hodnotu N budeme považovat hodnotu proudu naměřenou 
	mikroampérmetrem a za správnou hodnotu S hodnotu vypočtenou podle vztahu (1) vždy pro 
	tutéž hodnotu nastaveného napětí.
	
	\section{Naměřené hodnoty}
	\begin{tabular}{c||c|c|c|c|c}
		$U_1' [\si{\milli\volt}]$ & $\mu A_1 [\si{\micro\ampere}]$ & $\mu A_2 [\si{\micro\ampere}]$ & $\mu A_C [\si{\micro\ampere}]$ & $ V_2 [\si{\milli\volt}]$ & $I(V_2) [\si{\micro\ampere}]$ \\ \hline \hline
		20 & 0.6 & 0.5 & 0.91 & -97.1 & 0.971 \\ \hline
		40 & 1.4 & 1.5 & 2.57 & -274.1 & 2.741 \\ \hline
		60 & 2.4 & 3.0 & 5.58 & -597.2 & 5.972 \\ \hline
		80 & 3.8 & 5.0 & 10.73 & -1162.5 & 11.625 \\ \hline
		100 & 5.2 & 7.5 & 19.00 & -2089.7 & 20.897
	\end{tabular}
	\\\\\\
	\begin{tikzpicture}
		\begin{axis}[
			xlabel={$U_1' [\si{\milli\volt}]$},
			ylabel={$I [\si{\micro\ampere}]$]},
			xmin=18, xmax=102,
			ymin=0, ymax=21.5,
			xtick={20,40,60,80,100},
			ytick={0,2,4,6,8,10,12,14,16,18,20},
			legend pos=north west,
			xmajorgrids=true,
			ymajorgrids=true,
			grid style=dotted,
			]
			
			\addplot[
			color=red,
			mark=*,
			]
			coordinates {
				(20,0.6)(40,1.4)(60,2.4)(80,3.8)(100,5.2)
			};
			
			\addplot[
			color=blue,
			mark=*,
			]
			coordinates {
				(20,0.5)(40,1.5)(60,3.0)(80,5.0)(100,7.5)
			};
		
			\addplot[
			color=green,
			mark=*,
			]
			coordinates {
				(20,0.91)(40,2.57)(60,5.58)(80,10.73)(100,19.00)
			};
			
			\addplot[
			color=magenta,
			mark=*,
			]
			coordinates {
				(20,0.971)(40,2.741)(60,5.972)(80,11.625)(100,20.897)
			};
		
			\legend{$\mu A_1$,$\mu A_2$, $\mu A_C$, $I(V_2)$}
			
		\end{axis}
	\end{tikzpicture}
	
	\subsection{Výpočet vnitřních odporů}
	$
		I = 3 \si{\micro\ampere} = 0.003 \si{\milli\ampere}
	$
	
	\begin{tabular}{c||c|c}
		Přístroj & $U [\si{\milli\volt}]$ & $R [\si{\ohm}]$ \\ \hline \hline
		OZ & 42 & \\ \hline
		$\mu A_1$ & 43.5 & 500 \\ \hline
		$\mu A_2$ & 60 & 6000 \\ \hline
		$\mu A_C$ & 69 & 9000
	\end{tabular}
	
	\section{Výpočet nejistot}
	$$
	\begin{aligned}
		\Delta_{met}&=N-S\\
		\Delta_{met1}&=5.2-20.897=15.697\si{\micro\ampere}\\
		\Delta_{met2}&=7.5-20.897=13.397\si{\micro\ampere}\\
		\Delta_{metC}&=19.00-20.897=1.897\si{\micro\ampere}\\
		\\
		\sigma_{met}&=\frac{N-S}{S} \cdot 100\%\\
		\sigma_{met1}&=\frac{5.2-20.897}{20.897} \cdot 100\% = 75.116\%\\
		\sigma_{met2}&=\frac{7.5-20.897}{20.897} \cdot 100\% = 64.110\%\\
		\sigma_{metC}&=\frac{19.00-20.897}{20.897} \cdot 100\% = 9.078\%\\
	\end{aligned}
	$$
	\section{Závěr}
	Ukázali jsme, že čím vyšší je odpor ampérmetru, tím více exponenciálně roste naměřená hodnota. Zároveň jsme ukázali měření proudu pomocí zesilovače a voltmetru.
\end{document}