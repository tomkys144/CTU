\documentclass[titlepage]{article}
\usepackage[czech]{babel}
\usepackage{graphicx}
\usepackage{circuitikz}
\usepackage{amsmath}
\usepackage{subcaption}
\usepackage{pgfplots}
\usepackage{siunitx}
\usepackage{multirow}
\usepackage{pgfplots}
\usepackage{tabularx}
\usepackage{parskip}
\usepackage{float}
\sisetup{detect-all}
\begin{document}
	\begin{titlepage}
		\begin{center}
			\vspace*{1cm}
			
			\Large
			\textbf{LABORATORNÍ CVIČENÍ Z FYZIKY}
			
			\vspace{0.5cm}
			STUDIUM VOLNÉHO PÁDU
			
			\vspace{1.5cm}
			
			\textbf{Tomáš Kysela}\\
			\textbf{Skupina L1061}
			
			\vfill
			
			\vspace{0.8cm}
			
			
			ČESKÉ VYSKOKÉ UČENÍ TECHNICKÉ V PRAZE\\
			FAKULTA ELEKTROTECHNICKÁ\\
			KATEDRA FYZIKY\\
			30. 03. 2021
			
		\end{center}
	\end{titlepage}
	\section{Úvod}
	Objem válce dostaneme ze vztahu
	\begin{equation}
		V = \frac{\pi}{4} \cdot d^2 \cdot h
	\end{equation}
kde $r$ je poloměr a $h$ výška válce.
	Objem kvádru je poté
	\begin{equation}
		V = a \cdot b \cdot h
	\end{equation}
kde $a$ je šířka, $b$ je hloubka a $h$ výška kvádru.

Při výpočtu chyb použijeme následující vzorce:
\begin{equation}
	\bar{x} = \frac{1}{n}\sum_{i=1}^{n}x_i-\bar{x}
\end{equation}
\begin{equation}
	u_A(\bar{x}) = \sqrt{\frac{1}{n(n-1)}\sum_{i=1}^{n}(x_i-\bar{x})^2}
\end{equation}
kde $n$ je počet měření a $x$ měřená hodnota.
\begin{equation}
	u_B(\bar{x}) = \frac{R}{\sqrt{12}}
\end{equation}
kde R je rozlišení přístroje.
\begin{equation}
	u_C(\bar{x}) = \sqrt{u_A^2(\bar{x}) + u_B^2(\bar{x})}
\end{equation}

\section{Naměřené hodnoty}
\subsection{Válec}

\begin{figure}[h]
	\centering
\begin{tabular}{c|c|c}
	i & d [\si{\milli\meter}] & h [\si{\centi\meter}] \\\hline\hline
	1 & 12.90 & 3.216\\\hline
	2 & 12.89 & 3.210\\\hline
	3 & 12.87 & 3.206\\\hline
	4 & 12.81 & 3.208\\\hline
	5 & 12.80 & 3.2016\\\hline
	6 & 12.80 & 3.202\\\hline
	7 & 12.79 & 3.200\\\hline
	8 & 12.85 & 3.206\\\hline
	9 & 12.91 & 3.208\\\hline
	10 & 12.96 & 3.210
\end{tabular}
\end{figure}

$$
\begin{aligned}
	\bar{d} &= \frac{1}{10} \sum_{i=1}^{10}h_i = 12.858 \si{\milli\meter}\\
	u_A(\bar{d}) &= \sqrt{\frac{1}{90}\sum_{i=1}^{10}(h_i-12.858)^2} = 0.0182 \si{\milli\meter}\\
	u_B(\bar{d}) &= \frac{0.01}{\sqrt{12}} = 0.0029 \si{\milli\meter}\\
	u_C(\bar{d}) &= \sqrt{0.0182^2+0.0029} = 0.0184 \si{\milli\meter}\\
	\\
	\bar{h} &= \frac{1}{10} \sum_{i=1}^{10}d_i = 32.082 \si{\milli\meter}\\
	u_A(\bar{h}) &= \sqrt{\frac{1}{90}\sum_{i=1}^{10}(d_i-32.082)^2} = 0.0165 \si{\milli\meter}\\
	u_B(\bar{h}) &= \frac{0.2}{\sqrt{12}} = 0.0578 \si{\milli\meter}\\
	u_C(\bar{h}) &= \sqrt{0.0165^2+0.0578} = 0.0601 \si{\milli\meter}\\
	\\
	\bar{V} &= \frac{\pi}{4} \cdot 12.858^2 \cdot 32.082 = 416.5798 \si{\cubic\milli\meter}\\
	u(V) &= \sqrt{(\frac{\partial V}{\partial h}u_C(h))^2 + (\frac{\partial V}{\partial d}u_C(d))^2}\\
	u(V) &= \sqrt{(\frac{\pi d^2}{4} u_C(h))^2 + (\frac{\pi d h}{2} u_C(d))^2}\\
	u(V) &= \sqrt{(\frac{\pi 12.858^2}{4} \cdot 0.0601)^2 + (\frac{\pi 12.858 \cdot 32.082}{2} \cdot 0.0184)^2} = 14.2496 \si{\cubic\milli\meter}
\end{aligned}
$$

\subsection{Kvádr}

\begin{figure}[h]
	\centering
	\begin{tabular}{c|c|c|c}
		i & a [\si{\milli\meter}] & b [\si{\centi\meter}] & h [\si{\centi\meter}] \\\hline\hline
		1  & 18.04 & 3.02  & 3.268 \\ \hline
		2  & 18.08 & 3.022 & 3.266 \\ \hline
		3  & 17.94 & 3.01  & 3.276 \\ \hline
		4  & 17.74 & 3.018 & 3.27  \\ \hline
		5  & 18.31 & 3.008 & 3.264 \\ \hline
		6  & 17.96 & 3.006 & 3.26  \\ \hline
		7  & 17.32 & 3.016 & 3.262 \\ \hline
		8  & 16.62 & 3.018 & 3.268 \\ \hline
		9  & 19.19 & 3.01  & 3.266 \\ \hline
		10 & 18.66 & 3.024 & 3.276 \\
	\end{tabular}
\end{figure}

$$
\begin{aligned}
	\bar{a} &= \frac{1}{10} \sum_{i=1}^{10}h_i = 17.986 \si{\milli\meter}\\
	u_A(\bar{a}) &= \sqrt{\frac{1}{90}\sum_{i=1}^{10}(h_i-12.858)^2} = 0.2209 \si{\milli\meter}\\
	u_B(\bar{a}) &= \frac{0.01}{\sqrt{12}} = 0.0029 \si{\milli\meter}\\
	u_C(\bar{a}) &= \sqrt{0.2209^2+0.0029^2} = 0.2209 \si{\milli\meter}\\
	\\
	\bar{b} &= \frac{1}{10} \sum_{i=1}^{10}h_i = 30.152 \si{\milli\meter}\\
	u_A(\bar{b}) &= \sqrt{\frac{1}{90}\sum_{i=1}^{10}(h_i-12.858)^2} = 0.0198 \si{\milli\meter}\\
	u_B(\bar{b}) &= \frac{0.2}{\sqrt{12}} = 0.0578 \si{\milli\meter}\\
	u_C(\bar{b}) &= \sqrt{0.0198^2+0.0578^2} = 0.0611 \si{\milli\meter}\\
	\\
	\bar{h} &= \frac{1}{10} \sum_{i=1}^{10}h_i = 32.676 \si{\milli\meter}\\
	u_A(\bar{h}) &= \sqrt{\frac{1}{90}\sum_{i=1}^{10}(h_i-12.858)^2} = 0.0168 \si{\milli\meter}\\
	u_B(\bar{h}) &= \frac{0.2}{\sqrt{12}} = 0.0578 \si{\milli\meter}\\
	u_C(\bar{h}) &= \sqrt{0.0168^2+0.0578^2} = 0.0602 \si{\milli\meter}\\
	\\
	V &= a \cdot b \cdot h = 17.986 \cdot 30.152 \cdot 32.676 = 17720.6481 \si{\cubic\milli\meter}\\
	u(V) &= \sqrt{(\frac{\partial V}{\partial h}u_C(h))^2 + (\frac{\partial V}{\partial a}u_C(a))^2 + (\frac{\partial V}{\partial b}u_C(b))^2}\\
	u(V) &= \sqrt{(abu_C(h))^2 + (hbu_C(a))^2 + (ahu_C(b))^2}\\
	u(V) &= \sqrt{(17.986 \cdot 30.152 \cdot 0.0602)^2 + (32.676 \cdot 30.152 \cdot 0.2209)^2 + (17.986 \cdot 32.676 \cdot 0.0611)^2}\\ &= 222.986\si{\cubic\milli\meter}\\
\end{aligned}
$$

\section{Závěr}
Změřili jsme objem válce $(416.5798 \pm 14.2496) \si{\cubic\milli\meter}$ a kvádru $(17720.6481 \pm 222.986) \si{\cubic\milli\meter}$.
\end{document}