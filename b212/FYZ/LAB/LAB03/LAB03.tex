\def\tasknum{2}
\def\taskname{Čiselné soustavy}
\def\deadline{30. 09. 2022}

\def\in#1{_{(#1)}}
\def\to#1{\stackrel{(#1)}{=}}

\def\content{\begin{tasks}
	\task{\pair}

	Převeďte následující čísla mezi soustavami. Zdrojová soustava je zapsána v závorce za číslem, cílová nad znakem $=$.

	\begin{enumerate}
		\item $256\in{10} \to{2}$
		\item $1023\in{10} \to{2}$
		\item $1111\in{2} \to{10}$
		\item $0010111001\in{2} \to{10}$
		\item $16\in{10} + 1100\in{2} \to{10}$
		\item $16\in{10} + 1100\in{2} \to{2}$
	\end{enumerate}

	Za bonusové body můžete vyřešit ještě následující převody:

	\begin{enumerate}
		\item $FF\in{16} \to{10}$
		\item $15\in{6} \to{3}$
	\end{enumerate}

	\task{\pair}

	Rozmyslete si, co se stane s hodnotou čísla, pokud na konec jeho zápisu ve dvojkové soustavě napíšete $0$, případně $1$. Obdobně rozmyslete opačný proces: čemu odpovídá, pokud poslední číslici smažete?

	Za bonusové body otázku zodpovězte pro libovolnou poziční číselnou soustavu.

	\task{\pair}

	Mějme soustavu $a$. Jakou hodnotu má $10\in{a}$?

	\begin{notes}
		\note Nevíte-li si rady s takto obecně zadanou úlohou, zkuste si ji vyřešit pro konkrétní $a$. Převeďte číslo $10$ z dvojkové, trojkové, pětkové a šestnáctkové soustavy do desítkové a zkuste výsledek zobecnit. (Vymyslete, co platí pro všechny takové převody)
	\end{notes}

	\task{\pair}

	Mějme číselné soustavy $a$ a $b$. Která čísla se v nich zapisují shodně? Jinými slovy: znáte-li $a$ a $b$, co všechno můžete doplnit na místa tří teček tak, aby rovnost platila a napravo jste napsali to samé, co nalevo v $\ldots\in{a} = \ldots\in{b}$

	Nevíte-li si rady s takto obecně zadanou úlohou, dosaďte $a=9$ a $b=16$, tedy rozhodněte, která čísla se v devítkové a šestnáctkové soustavě píší shodně. Výsledek nemusíte zobecnit.

	Bonus: Která čísla se zapisují shodně ve všech pozičních soustavách?

\end{tasks}}
